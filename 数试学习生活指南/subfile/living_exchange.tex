\section{出国交流}

\subsection{交流项目介绍}
    这里所列举的交流项目是目前已经确定的交流项目,仅供参考,具体若有改动,还以学院通知为准。
\begin{table}
    \begin{tabular}{c|p{3cm}|c|p{3cm}|p{3cm}|p{3cm}}
        \hline
        \hline
        年级  &  学校  & 时间  &  优点  &  缺点  &  费用\\ \hline
        \tabincell{c}{ \\\\大一\\大二} & \tabincell{c}{ 加拿大\\阿尔伯塔大学\\(10人)} & 7月初-8月初  
                & 可以提前适应西方的学习生活,为大三的交流作准备 & 时间较短,不容易沉下心学习 & 除去报销费用后,个人花费一般不到一万人民币\\ \hline
        大三下 & 新加坡国立大学NUS(3-5人) & 1月初到5月初 & 学校排名较高,有安排导师 &  &   \\ \hline
                & 密歇根州立大学MSU(10-15人) &     & 数学专业较强,数学课全,安排导师,接待周到,费用低 & 学校排名相对较低  &  \\ \hline
                & 佐治亚理工大学GT(10-15人) &     &动力系统方向较强,有安排导师 & 住宿环境可能不太好 &  \\ \hline
                & 加州大学伯克利分校UCB(3-5人) &   & 学校排名较高,数学专业强势 & 很难找导师,费用高 & \\ 
                \hline
                \hline
    \end{tabular}
    \caption{交流项目简介}
    \label{ex-pro}
\end{table}


\subsection{材料}
\begin{enumerate}
    \item \Emph{护照:}应尽早办好,有效期十年;需要本人到当地区县级以上公安局办理,需要十个工作日。
    \item \Emph{visa信用卡副卡:}在银行办理以父母信用卡为主卡的副卡,需要本人。办副卡是最理想的方案,
                        直接用主卡可能会出问题(非持卡人刷信用卡可能会遭到拒绝),但如果实在来不及办副卡,
                        建议在你在信用卡背面签上你父母(持卡人)的名字,刷卡需要签名时,也签上你父母的名字即可。
    \item \Emph{签证:}确定交流名单后,即可着手办理,其中,签证需要材料如下:
        \begin{enumerate}[(1)]
            \item \Emph{护照}。
            \item \Emph{财产证明:}银行六个月内流水(不宜过多也不宜过少,建议专门开一个账户,存款一万-十五万);行驶证;房产证明。
            \item \Emph{家庭成员身份信息:}户口本复印件(如与父母不在一个户口本上,应出示出生证明)、身份证复印件。
            \item \Emph{父母在职证明:}加盖单位公章。
            \item \Emph{父母单位营业执照复印件}。
            \item \Emph{校方邀请信:}学院会发。
            \item \Emph{与护照一致的照片:}一致指的是书否戴眼镜、是否扎头发。建议在公安局拍护照照片时,带上u盘拷电子版,
                        以后需要时打印照片即可;当时没有拷的话,也可提前抽空拍一张一致的照片,同时索要电子版,
                        以后需要证件照时只需打印,无需费力找照相馆。
            \item \Emph{签证资料表:}在大使馆官网上可下载。
        \end{enumerate}
    \item \Emph{保险:}签证办好后,便可联系买保险的事宜。
\end{enumerate}

\subsection{出国应携带的物品}
    \begin{enumerate}
        \item \Emph{证件:}护照、身份证(非必需)、邀请信(电子版即可,打印也行)、签证(在护照内)、学生证、机票。
        \item \Emph{随身品:}少量当地货币(非必需)、visa信用卡、银联卡(在具有银联标识的atm机上可直接取现金)、路上的零食、纸巾、水杯。
        \item \Emph{电子产品:}手机、电脑、相机、各种线、耳机、鼠标、当地电源转换器(加拿大和美国都是美标)、U盘、移动电源。
        \item \Emph{学习用品:}书、笔、纸、本子。
        \item \Emph{生活用品:}衣物、洗漱用品、毛巾、拖鞋。
        \item \Emph{乱七八糟:}自拍杆(合影神器)、电池、泳衣泳镜。
        \item \Emph{厨房用品:}酱油、蚝油、调料盒、盐(米可以不用买,至少加拿大的米不贵;这些东西可以一伙人各买一点)、电饭煲\&变压器
                    (手机、电脑的插头自带变压,但电饭煲的不一定自带变压,如果没有一定要带变压器,不然米煮不熟)、保鲜膜。
    \end{enumerate}