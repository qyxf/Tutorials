\section{教材目录}
为方便学有余力的同学自学(实际上我相信每一个同学如果认真学习,也就是能做到上课认真听讲、课后认真复习、课后习题都做一遍的话,都可以做到学有余力),我们在这列出2017级数试大一到大三的数学课程教材目录。

    \begin{table}[h]   
    \begin{tabular}{c|c|c|c|c|l}
        %\multicolumn{5}{c}{16级大一大二教材目录}\\
        \hline
        \hline
        学期 & 课程名 & 教材名 & 作者 & 出版社 & 备注\\
        \hline
        大一  & 数学分析 & 数学分析教程   & 常庚哲、史济怀 & 高等教育出版社  &  \\ \hline
              & 高等代数 & 高等代数	& 丘维声	&科学出版社	& \\ \hline
              & 数论基础 & 初等数论	& 闵嗣鹤	&高等教育出版社&  \\ \hline	
        大二上	&数学分析	&数学分析教程	&	&	&多元微积分  \\ \hline
               &常微分方程	&常微分方程及其应用	&周义仓	&科学出版社&\\ \hline
               &近世代数	&代数学引论	&聂灵沼, 丁石孙	&北大出版社&\\ \hline
        大二下	 &数值分析	&数值线性代数&  徐树方	&	北大出版社&\\ \hline
                &        &数值逼近	& 蒋尔雄	&复旦出版社&\\ \hline
                &拓扑学&	基础拓扑学讲义	&尤承业	&北大出版社& \tiny{点集拓扑+基本群} \\  \hline
                &实变函数	&实变函数论	&周民强	&北大出版社&\\ \hline
                &复变函数	&Complex Analysis&	Stein &	&\\ \hline
                &概率论	&概率论	&苏淳	&科学出版社&\\ \hline     
        大三上  &泛函分析  &泛函分析讲义(上册) &张恭庆 &北大出版社 &\\ \hline
                & 微分几何  & 微分几何初步 & 陈维桓 &北大出版社 &\\ \hline
                &偏微分方程 & 数学物理方程讲义 &姜礼尚 & 高等教育出版社 &\\ \hline  
        \hline
    \end{tabular}  
    \caption{17级大一到大三数学教材目录}
    \end{table}  
