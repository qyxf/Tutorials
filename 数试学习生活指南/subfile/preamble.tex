    \newsavebox{\mybox}
    \usepackage{titlesec}%定制section的样式
    \usepackage{geometry}
    \usepackage{cite}
    
    \usepackage{varwidth}
    \usepackage[thmmarks]{ntheorem}
    \usepackage{float}%优化插图位置
    \usepackage{wrapfig}%插入图片使之被文字环绕
    \usepackage{multirow}%表内多行合并
    \usepackage{array}%表格内自动换行命令所需 \begin{tabular}{m{5cm}}
    
    % {
    %  \theoremheaderfont{\bfseries}
    %  \theorembodyfont{\normalfont}
    %  \newtheorem*{pf}{证}
     
    % }
    % {
    %  \theoremheaderfont{\bfseries}
    %  \theorembodyfont{\normalfont}
    %  \newtheorem{example}{例}
     
    % }
    \usepackage{enumerate}%定制enumerate环境
    %\usepackage{enumitem}%定制enumerate环境的序号样式
    %\AddEnumerateCounter{\chinese}{\chinese}{}%增加中文序号样式到enumerate环境,使用示例:
                               %\begin{enumerate}[label={\chinese*、},labelsep=0pt]
    \usepackage{extarrows}
    \usepackage{mathtools}
    \usepackage{amsfonts}
    \usepackage{color}
    \usepackage{float}
    \usepackage{geometry}
    \usepackage{theorem}
    \usepackage{amsmath}
    \usepackage{amssymb}
    \usepackage{ulem}%记号
    \usepackage{bm}
   % \newtheorem{definition}{定义}
    \newtheorem{thm}{定理}
 %   \newtheorem{lemma}{引理}
    \newtheorem{prop}{性质}
    \newtheorem{method}{解法}
    \newtheorem{notice}{注}
    %\newtheorem{example}{例}
    %\newtheorem{proof}{证}
    \newtheorem{cor}{推论}
    
    
    \newcommand{\tabincell}[2]{\begin{tabular}{@{}#1@{}}#2\end{tabular}}%使表格内可手动换行
    \newcommand{\mylim}{\lim\limits_{n\to\infty}}
    \newcommand\Emph{\textbf}
    \newcommand\degree{^\circ}
    \newcommand\red{\color{red}}
    \newcommand\ue{\mathrm{e}}%e的正体版
    \newcommand\dif{\mathrm{d}}%dx正体版.
    \newcommand\diff{\mathrm{D}}%D正体版.
    \newcommand\mx{\boldsymbol}%粗体.
    \makeatletter
    \newcommand{\rmnum}[1]{\romannumeral#1}
    \newcommand{\Rmnum}[1]{\expandafter\@slowromancap\romannumeral#1@}
    \newcommand{\tr}{\mathop{\mathrm{tr}}}%\trace 
    \makeatother
    
    \bibliographystyle{plain}
    
    \renewcommand\qedsymbol{\ensuremath{\Box}}
    %\renewcommand\refname{参考文献}
    
    \usepackage[pdftex, bookmarksnumbered, bookmarksopen, colorlinks, citecolor=blue, linkcolor=blue]{hyperref}%定制目录样式
 %   \geometry{a4paper,margin=1in}
  %  \pagestyle{plain}