\section{学习态度}
首先,你应当去思考,来大学是为了什么,学数学又是为了什么,现在肯定还没有答案,但这两个问题希望贯穿你四年的生活,让你不断有更好的答案,当你可以给出更精彩的答案时,你的生活,你自己都会变得更加绚丽。但我要提醒的是,不要轻视或者简化这个问题。当选择数学试验班的时候,有人是立志成为数学家;有人是因为觉得数学有趣;有人觉得数学是万金油,学好了做任何事情都可以;有人因为没什么专业喜欢,先学个数学以后好转行;或者有人是因为数学试验班资源很好,有出国交流的机会,重点培养而来。我想这所有的答案都可以作为选择的依据,但不能作为度过这四年的理由。你会在此经历学业上的困难,生活上的挫折,来自未来越来越多的压力。立志成为数学家的同学可能发现数学研究并不是自己所想的,而想转行的同学可能会发现科研可能是最适合自己的职业,觉得数学有趣的同学可能会失去兴致,但对数学知识感觉一般的同学可能渐渐发现数学的惊奇。但你只有不断思考自己想在大学学到什么,想从数学中学到什么,你才能迎难而上,创造出自己的大学生活。所以学习态度不是一个别人给予的答案,不是一个我感兴趣,我不感兴趣的答案,但我依然想做一些鼓励和提醒来帮助大家去树立自己的学习态度,因为这在我们回顾已经度过的大学时光后一致认为是最为重要的。

\subsection{什么是专业——数学外的学习}
大家的专业是数学尤其是作为数学试验班的学生,学好数学自然是学习中最重要的一环,但大学学习远不止这些,这种环境,可以让你去接触一切你感兴趣的东西,经济,历史,哲学,心理,生物,物理,宇宙学,工学,舞蹈,排球等等的内容,虽然你可以从饱满的学习任务中挤出的时间很少,但我仍鼓励你可以在不影响自己主课学习的过程中去经历不同的课或者活动,结交不同的朋友。其实我们有这样的时间和机会,但如果你每次选课都在思考哪个给分高而不是了解一下具体的内涵,只在意GPA而不是自己的成长,那或许没有这样的机会和时间。所以专业是选了一个主旋律,但如果没有别的东西,那称不上精彩。

\subsection{数学的学习态度}
很奇怪的是,我先提及了非数学的学习,因为数学的学习绝对是一个复杂且具有挑战性的事情,你应当在其中有认真的态度,你应当掌握课内要求的知识,做充足的练习并且对数学积累足够的经验和感觉,训练有素是个基本的目标。但我想提醒的是,虽然我们是理科,学习内容,学习方式都是,但数学本身却又是个十分感性的学科,我希望你可以时常拿文学,哲学来类比。学文学肯定不会是去背诵文章主旨,起码大学不是,那学数学也不是单纯掌握和应用数学书上的定理和命题。你应当追求一种体系的建立,以自己的方式梳理你学过的每一门课和每一门之间的联系,这当然是在掌握课内知识的基础上,但这才是最重要的一步,因为数学是为了提高人们理解和描述概念和问题的能力,给你新的看待世界的方式,塑造你无与伦比的思维模式,而这些正是通过定理和命题这种抽象的方式而进行的,所以你需要时刻惦记着这件事。除此之外,数学的学习和上一条其他方面的学习是紧密相连的,我们通过学习数学,事实上应当学习到教育所期待我们学习的一切,我希望你们可以在学习过程中感受到,学数学不是修一门课,读一本书,证明一个命题。她是一个完整的思维奇迹,一门感性的艺术。学习数学是让抽象概念,思维方式,融入大脑的过程。这些概念会反复挑战我们的思维,进而来帮助自己更好的理解这个世界,无论是生物,物理,工程,计算机,还是经济,社会体系,甚至是哲学,心理学。数学,尤其大学里学习数学,远不是今后非要拿这些理论来谋生,即便是数学工作者,用到的只是某一专一领域中的知识。这应是一场旅程,一次受人类智慧熏陶的过程,希望数学的智慧,可以鼓舞你不断思考,反复追问,帮助你们在四年中去思考,自己将成为什么样的人,过怎样的生活。希望你可以沉浸在这样的氛围中,去结交朋友,思考问题,碰撞思维,因为这就是数学的奇妙,她让你碰到无数有趣的灵魂,难忘的经历,带来无与伦比的体验,最终引领我们自我实现。所以,当你认真对待这份值得你精力的学业,我坚信,生活的美好将会在面前自然地展开,祝你们好运!


