%!TEX program = xelatex
\documentclass[a4paper,fleqn,twocolumn]{ctexart}
\usepackage{xeCJK}
\usepackage{geometry}
\geometry{left=2.0cm,right=2.0cm,top=2.5cm,bottom=2.5cm}
\usepackage{amsmath}
\usepackage{amssymb}
\usepackage{graphicx}
\setCJKmainfont{SimSun}
\newcommand{\di}[1]{\mathrm{d}#1}%积分式的d
\newcommand{\p}[2]{\dfrac{\partial #1}{\partial #2}}%一阶偏导
\newcommand{\pp}[2]{\dfrac{\partial ^2 #1}{\partial #2 ^2}}%二阶偏导
\newcommand{\dy}[2]{\dfrac{\di{#1}}{\di{#2}}}%一阶导数
\newcommand{\ddy}[2]{\dfrac{\mathrm{d} ^2 #1}{\mathrm{d} #2 ^2}}
\linespread{1.2}
\begin{document}
	\section*{第一章}
		\noindent
		一、选择题\\
		1.A\\
		\indent
		如图1.1,对M,在x方向上:
		\vspace{-1em}
		\begin{figure}[htbp]
			\centering
			\includegraphics[width=10em,height=15em]{c1_1.png}
			\quad
			\centering
			\includegraphics[width=10em,height=15em]{c1_2.png}
		\end{figure}
		\vspace{-3.5em}
		\begin{gather}
		N\sin\theta=Ma_e\text{(M对地)}\\
		\text{如图1.2,对m,以M为参考系,m受一惯性}\notag\\
		\text{力,合加速度沿二者接触面。沿x,y方向分解:}\notag\\
		mg-N\cos\theta=ma_r\sin\theta\\
		ma_e+N\sin\theta=ma_r\cos\theta\\
		\text{(1)代入(2),(2)(3)联立解得:}\notag\\
		a_r=\dfrac{(M+m)g\sin\theta}{M+m{\sin\theta}^2}\notag
		\end{gather}
		2.B\\ \indent
		如图1.3,$u=v\cos\theta$,v不变而$\theta$增大,需要u减小。\\
		\vspace{-3.5em}
		\begin{figure}[htbp]
			\centering
			\includegraphics[width=6.5em,height=10em]{c1_3.png}
			%\caption{}
			\label{fig:c1_2}
		\end{figure}
		\vspace{-4.5em} \\
		3.A\par
		匀速圆周运动的速度、加速度(受力)均是大小不变、方向时刻变化。
		注意一个矢量为常量包括大小和方向两个方面。否则就是变化的量。\\
		4.B\par
		以前面的货车为参考系,货车静止,火车速率为$v_1-v_2$,加速度为$a$(反向),那么火车最多前进$s=\dfrac{{(v_1-v_2)}^2}{2a}$。要求$d>s$,故选B。或采用地面参考系的追逐问题法,计算从$v_1\text{减速到}v_2$两车走过的距离之差:$s=\dfrac{{v_1}^2-{v_2}^2}{2a}-v_2\cdot \dfrac{v_1-v_2}{a}=\dfrac{{(v_1-v_2)}^2}{2a}$\\
		5.C\par
		两次求导得:$a=30t\neq$常数而大于零。\\
		6.B\par
		求导得:$v=8t-6t^2,a=8-12t$\par
		$\text{令}y=0\Rightarrow t=0\text{(舍去)或}2$,代入得结果。\\
		7.B\par
		物体做匀加速直线运动。\vspace{-1em}
		\begin{gather*}
		s=\dfrac{b}{\cos\alpha},a=g\sin\alpha\\
		t=\sqrt{\dfrac{2s}{a}}=\sqrt{\dfrac{4b}{g\sin(2\alpha)}}
		\end{gather*}
		\par $t$最小时,$\sin(2\alpha)$最大,$\alpha=45^\circ$。\\
		8.B\par 
		类比从静止出发的匀加速直线运动。$t=\dfrac{2\cdot 2\pi}{\beta}$\\
		9.B\par 
		曲线的定义:“动点运动方向连续变化的轨迹”\footnote{来源:汉典网http://www.zdic.net/c/2/111/299079.htm}。A,C的反例:匀速圆周运动。\\
		10.D\par 
		反例:平抛运动\\		
		二、填空题\\
		11.\ 0\qquad2g\\
		\indent
		设A、B质量为m。抽走C之前,弹簧中的弹力大小为mg。撤去C时,弹簧长度未突变,弹力不变,A受合力为0;支持力则消失。\par
		故$a_A=0;a_B=\dfrac{mg+mg}{m}=2g$,竖直向下。\\
		12.\ $\dfrac{25}{12}\pi\ rad/s^2$ \qquad$\dfrac{24}{5}s$
		\begin{gather*}
		\text{简单公式应用。}\theta=60\times2\pi=120\pi\\
		\beta=\dfrac{\omega_2^2-\omega_1^2}{2\theta}=\dfrac{25}{12}\pi\ rad/s^2\\
		\delta t=\dfrac{\omega_2-\omega_1}{\beta}=\dfrac{24}{5}s
		\end{gather*}
		13. $m(\sin\theta-\omega^2l\sin\theta\cos\theta)$
		\vspace{-1em}
		\begin{figure}[htbp]
			\centering
			\includegraphics[width=15em, height=15em]{c1_4.png}
			%\caption{}
			\label{fig:c1}
		\end{figure}
		\vspace{-3em}
		\begin{gather*}
			\text{如图1.4,在x,y方向上分解受力,得:}\\		
			T\sin\theta-N\cos\theta=m\omega^2l\sin\theta\\
			T\cos\theta+N\sin\theta=mg\\
			\text{联立可解得T、N的大小。}
		\end{gather*}
		14.\ $2\sqrt{\dfrac{r}{g}}$ \qquad $2\sqrt{\dfrac{r}{g}}$
		\vspace{-0.5em}
		\begin{gather*}
		\text{设弦与PC的夹角为}\theta,\text{则有}\\
		s=2r\cos\theta,a=g\cos\theta\\
		t=\sqrt{\dfrac{2s}{a}}=2\sqrt{\dfrac{r}{g}}
		\end{gather*}
		15.\ $4\sqrt{5}m/s$ \qquad $16m/s^2 $ 
		\vspace{-1em}
		\begin{gather*}	
		\text{抛物线的切线方向即为质点的速度方向,且}x=4t\\
		\therefore \dfrac{v_y}{v_x}=\dy{y}{x}=x\Rightarrow v_y=4x=16t\\
		\therefore t=2\text{时,}v_y=8\mathrm{m/s},
		v=\sqrt{{v_x}^2+{v_y}^2}=4\sqrt{5}\mathrm{m/s}\\
		v_x\text{不变,}a=a_y=\dy{v_y}{t}=16\mathrm{m/s^2}
		\end{gather*}
		16. 长度、质量、时间\par
		见课本,解答略。\\
		17. 3\quad 3\quad 6\par
		x分别对t求一阶两阶导即是v、a,由图像即可判断其正负号。\\
		18. $y={(x+5)}^3$
		\vspace{-1em}
		\begin{gather*}\text{由题,}x=2t-5\Rightarrow 2t=x+5\\
		\text{代入}y=8t^3={(2t)}^3\text{,消去t即可}
		\end{gather*}	
		19. $\dfrac{1}{2}g$\qquad 竖直向下\par
		初始时受力平衡,两根弹簧上力均为$\dfrac{1}{2}mg$;一根断掉后,向上的力减半,则小球受的合力是$\dfrac{1}{2}mg$,竖直向下。\\
		20.\ 9m/s 
		\vspace{-1em}
		\begin{gather*}\text{(SI)} x=3t+6t^2-2t^3\xrightarrow{\text{求导}}v=3+12t-6t^2\\
		\xrightarrow{\text{求导}}a=12-12t\\\
		令a=0\ \text{解得}\ t=1\xrightarrow{\text{代入得}} v(1)=9\mathrm{m/s}
		\end{gather*}
		三、计算题\\
		%21,22,24题对原作者代码有改动
		21.
		\begin{gather*}
			\text{由图知:}\\
			\tan\alpha=\dfrac{|\vec{a}_n|}{|\vec{a}_\tau|}=\dfrac{\dfrac{v^2}{R}}{\dy{v}{t}}\\
			\therefore \dy{v}{t}\frac{1}{v^2}=\frac{1}{R\tan\alpha}\\
			\text{积分得:}-\frac{1}{v}=\frac{1}{R\tan\alpha}t+C\\
			\text{代入}t=0,v=v_0\\
			\therefore \frac{1}{v_0}-\frac{1}{v}=\frac{1}{R\tan\alpha}t
			\ \text{即}v=\frac{v_0R\tan\alpha}{\tan\alpha-v_0t}
		\end{gather*}
		22.
		\begin{gather*}
			\vec{v}=\dy{s}{t}\vec{\tau}=(c+2dt)\vec{\tau}\\  
			\vec{a}_n=\frac{v^2}{R}\vec{n}=\frac{(c+2dt)^2}{R}\vec{n}\\
			\vec{a}_\tau=\dy{v}{t}\vec{\tau}=2d\vec{\tau}\\
			\text{令}|\vec{a}_n|=|\vec{a}_\tau|,
			\text{则}\frac{(c+2dt)^2}{R}=2d\\
			\therefore t_1=\frac{\sqrt{2dR}-c}{2d}\left(t_2=\frac{-\sqrt{2dR}-c}{2d}<0\text{,舍去}\right)\\
			\therefore\text{要使t}\geqslant\text{0,条件为}\sqrt{2dR}-c\geqslant0,\text{即}2dR\leqslant c^2
		\end{gather*}
		23.
		\begin{gather*}
			-kx=a=\dy{v}{t}=\dy{v}{x}\cdot \dy{x}{t}=\dy{v}{x}\cdot v\\
			\text{分离变量,积分得:}-\dfrac{1}{2}kx^2=\dfrac{1}{2}v^2+C_1\\
			\text{令}C=2C_1,\text{则}-kx^2=v^2+C\\
			\text{代入}x=x_0,v=v_0\text{\,得:}C=-(kx_0^2+v_0^2)\\
			\text{整理得:}v=\pm\sqrt{kx_0^2+v_0^2-kx^2}
		\end{gather*}
		24.
		\begin{align*}
		(1)v&=10\left(1-\frac{t}{5}\right)\\
		&=-2t+10\\
		\dy{x}{t}&=-2t+10
		\end{align*}
		\vspace{-2.5em}
		\begin{gather*}
		\text{积分得:}x=-t^2+10t+c\\
		\text{代入}t=0,x=0,\quad \therefore x=-t^2+10t\\
		\text{代入}t=10s\\
		\therefore x=0.\quad
		\therefore\text{坐标为}0\\
		(2)\text{令}x=10m,\therefore t^2-10t+10=0\\
		\therefore t=5\pm\sqrt{15}s\\
		\text{令}x=-10m,\therefore t^2-10t+10=0\\
		\therefore t=5+\sqrt{35}(5-\sqrt{35}<0,\text{舍去})\\
		\therefore\text{时刻为}5-\sqrt{15}\ s,5+\sqrt{15}\ s\text{或}5+\sqrt{35}\,s\\
		(3)\text{令}v=0,\therefore t=\ 5s\\
		\therefore t\in[0,5],\quad s=x=-t^2+10t\\
		t\in[5,+\infty),\quad s=s(5)+[s(5)-x]=25\\+[25-(-t^2+10t)]
		=t^2-10t+50\\
		\therefore s=
		\begin{cases}
		-t^2+10t,&t\in[0,5)\\
		t^2-10t+50,&t\in[5,+\infty)
		\end{cases}
		\end{gather*}
\end{document}
